% !TeX root = ../pythonTutorial.tex
\chapter{Fazit}

Wie schon aus verschiedenen anderen Projekten hat die Zusammenarbeit problemlos funktioniert und Absprachen sowie regelm��ige Treffen im Virtual Reality Labor wurden �ber den gesamten Projektverlauf hinweg gepflegt. Alle Probleme waren nach und nach gut l�sbar und es wurde innerhalb des Projektverlaufes einiges in gegenseitigem Einverst�ndnis ge�ndert, was jedoch im Nachhinein immer eine gute Entscheidung war. Der einzige Wermutstropfen ist der zu kurz gekommene Android Build, f�r den am Ende die Zeit ausging. Zwar ist eine Ausf�hrung �ber Android Ger�te wie die Samsung VR Gear prinzipiell m�glich, allerdings ist die Steuerung der HTC Vive Pro nicht kompatibel mit solchen Low Budget Modellen. Diese Aufgabenstellung, sowie eine Evaluierung im Schul- oder Hochschulumfeld sind Dinge, die f�r die Zukunft des Projektes denkbar sind und einen n�chsten Schritt bilden sollten. \newline

Wir sind sehr froh, dass wir nun �ber einen Zeitraum von fast einem Jahr (mit dem vorherigen Semester zusammen) an dem Projekt arbeiten durften, was uns einen tiefen Einblick in die Entwicklung von Virtual Reality Anwendungen, vor allem in der Entwicklungsumgebung Unity vermitteln konnte. Auch die Erfahrungen im Arbeiten mit Git konnte deutlich verbessert werden, was mittlerweile kein Problem mehr f�r einen guten Arbeitsfluss aufgeworfen hat.  \newline

Das alleinige Ausarbeiten von Ideen, unterst�tzt von Herr Prof. Dr. Manfred Brill konnte uns jedoch vor allem tiefe Einblicke bieten, wie es ist, an einem dynamischen Projekt von  der Planung bis zur Entwicklung teilzuhaben, was uns f�r unsere sp�tere Karriere noch von gro�em Nutzen sein kann. Dabei waren auf die aufgetretenen Fehler und Probleme eine gro�e Hilfe, denn die Realit�t der Projektdurchf�hrung ist (wohl fast) nie wie die Vorstellung und Planung. \newline

Abschlie�end gilt unser Dank Herrn Prof. Dr. Brill, sowie Fabian Kalweit, die auch kurzfristig stets Zeit gefunden haben, um mit uns den weiteren Ablauf zu besprechen und auch etliche Ideen mit auf den Weg gegeben haben. Au�erdem sind wir sehr dankbar f�r die Modelle des Mikroskops und des Bunsenbrenners unseres Kommilitonen Cedric Schug, der falls der Unity Asset Store nicht ergiebig war unter die Arme gegriffen hat. Zu guter Letzt auch noch danke an Herr Dr.-Ing. Hubert Zitt, der sich ebenfalls viel Zeit nahm, um uns die beiden Elektrotechnik-Versuche genau aufzubauen, zu zeigen und zu erkl�ren. 