% !TeX root = ../pythonTutorial.tex
\chapter*{Vorwort}

Der vorliegende Projektbericht gibt einen �berblick �ber das entworfene Virtual Science Lab. Dabei handelt es sich um eine Virtual Reality Anwendung, die dem Benutzer intuitiv Versuche aus verschiedenen Wissenschaftszweigen veranschaulichen und erkl�ren soll. Als Programmierumgebung wurde hierzu Unity gew�hlt, die spezifische Hardware, f�r die die Anwendung optimiert wurde ist die HTC Vive Pro. Die bereits vorhandene Version von den Autoren und Anatoli Sch�fer wurde um einige weitere Labore f�r Elektrotechnik, Biologie, Mathematik und Informatik, zus�tzlich zu den bereits vertretenen Chemie- und Physiklaboren erg�nzt.

Durch diese Erweiterung offeriert die Anwendung eine gr��ere Anzahl an verschiedenen Versuchen und es besteht die Annahme, dass durch das spielerische und native Ausprobieren, der Nutzer l�nger bereit ist, sich indirekt �ber Wissenschaft zu informieren. 

Ein Ausblick in die Zukunft zeigt, dass diese Annahme in Tests ausgewertet werden muss und dass eine Steuerung f�r verschiedenste Hardware-Varianten evaluiert werden muss. 