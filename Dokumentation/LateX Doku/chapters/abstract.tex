% !TeX root = ../pythonTutorial.tex
\chapter*{Vorwort}

Der vorliegende Projektbericht gibt einen Überblick über das entworfene Virtual Science Lab. Dabei handelt es sich um eine Virtual Reality Anwendung, die dem Benutzer intuitiv Versuche aus verschiedenen Wissenschaftszweigen veranschaulichen und erklären soll. Als Programmierumgebung wurde hierzu Unity gewählt, die spezifische Hardware, für die die Anwendung optimiert wurde ist die HTC Vive Pro. Die bereits vorhandene Version von den Autoren und Anatoli Schäfer wurde um einige weitere Labore für Elektrotechnik, Biologie, Mathematik und Informatik, zusätzlich zu den bereits vertretenen Chemie- und Physiklaboren ergänzt.

Durch diese Erweiterung offeriert die Anwendung eine größere Anzahl an verschiedenen Versuchen und es besteht die Annahme, dass durch das spielerische und native Ausprobieren, der Nutzer länger bereit ist, sich indirekt über Wissenschaft zu informieren. 

Ein Ausblick in die Zukunft zeigt, dass diese Annahme in Tests ausgewertet werden muss und dass eine Steuerung für verschiedenste Hardware-Varianten evaluiert werden muss. 