\subsection{Daten laden}\label{maschinelleslernen:datenladen}
Es gibt verschiedene Herangehensweisen, meist bietet es sich an, erst mal einen groben �berblick �ber die Daten zu erhalten. F�r die erste Erl�uterung werden Datens�tze angenommen, welche in folgender Datenstruktur vorliegen.

\subsubsection*{Daten aus Datei lesen}
Maschinelles Lernen \randnotiz{Beispieldaten} macht nur Sinn mit entsprechenden Daten. Diese sollten im Besten Fall bereits in einer strukturierten Form vorliegen.
\lstinputlisting[language=Python]{chapters/advancedTopics/src/machinelearning/datasample1.txt}\label{datasample1:lst:datasample}


Nun kann mit der Entwicklung begonnen werden. Um den Datentyp \randnotiz{Datei auslesen} mit Daten zu versorgen findet sich im Folgenden ein kleiner Codeausschnitt:
\lstinputlisting{chapters/advancedTopics/src/machinelearning/readdatafromfile.py}\label{readdatafromfile:lst:readdata}

Die aufbereiteten Daten k�nnen im n�chsten Schritt visualisiert werden. Vorher noch zwei weitere Varianten wie Daten geladen werden k�nnen.

\subsubsection*{Daten aus Paket laden}
Eine\randnotiz{iris-Dataset laden} weitere M�glichkeit ist das Laden von Daten aus Paketen. Hier wird beispielhaft das Lesen aus dem Paket \lstinline$iris$ vorgestellt.
\lstinputlisting[language=Python,firstline=3,lastline=19]{chapters/advancedTopics/src/machinelearning/loadiris.py}\label{loadiris:lst:loadiris}

Die Ausgabe f�r \lstinline$target_names$ ist folgende:
\begin{lstlisting}
[0 0 1]
['setosa', 'versicolor', 'virginica']
\end{lstlisting}


\subsubsection*{Daten aus URL laden}
Eine\randnotiz{iris-Dataset aus URL laden} weitere M�glichkeit ist das Laden von Daten aus einer URL. Hier wird als Beispiel wieder das Paket \lstinline$iris$ genommen.
\lstinputlisting[language=Python]{chapters/advancedTopics/src/machinelearning/loadirisurl.py}\label{loadiris:lst:loadirisurl}

\uebung
\aufgabe{MachineLearning/machinelearning_datasets}