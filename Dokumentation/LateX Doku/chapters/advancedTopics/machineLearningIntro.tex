\section{Maschinelles Lernen in Python}\label{maschinelleslernen:einleitung}
Das Themengebiet des maschinellen Lernens kann verschiedene Komplexit�tslevel erreichen. Grundlegend ist das mathematische Verst�ndnis �ber die verschiedenen im maschinellen lernen eingesetzten Algorithmen. Diese werden in diesem Tutorial nicht beschrieben. \\
Sind die Algorithmen bekannt und sollen nun mittels Python angewendet werden, sollte zu Beginn mit einem kleinen Projekt gestartet werden. Hierbei ist es sinnvoll sich bereits am Anfang eine Vorgehensweise zu �berlegen, wie auch bei allen anderen Projekten. Python bietet im Bereich des maschinellen Lernens viele unterschiedliche M�glichkeiten an, sodass bereits fr�hzeitig der Aufbau des Projekts entschieden werden sollte. Grob kann ein Projekt in f�nf Schritte aufgeteilt werden, anhand derer sp�ter das Ergebnis verifiziert werden kann. \\

\begin{enumerate}
\item zu l�sendes Problem definieren
\item Daten verstehen und vorbereiten
\item m�gliche Algorithmen evaluieren
\item Ergebnisse verbessern
\item Ergebnisse darstellen
\end{enumerate}


Eine weit verbreitete M�glichkeit ist das Einbinden von bereits existierenden Bibliotheken, die viele Funktionalit�ten von Haus aus anbieten. Eine eigene Nachbildung von verbreiteten Algorithmen aus dem Bereich Maschinelles Lernen ist daher meist nicht n�tig.\\
Eine zweite M�glichkeit ist die Integration von R. Bei R handelt es sich um eine eigene Programmiersprache, welche den Schwerpunkt in mathematischen Probleml�sungen hat. Python und R lassen sich beide sowohl eigenst�ndig, also auch in Verbindung miteinander einsetzen.

Im Folgenden Abschnitt gibt es eine �bersicht, �ber wichtige und bekannte Bibliotheken aus dem Bereich maschinelles Lernen. Die Anzahl der Bibliotheken macht den Einstieg nicht ganz leicht. Die St�rken und Schw�chen der einzelnen Bibliotheken sollten betrachtet werden.