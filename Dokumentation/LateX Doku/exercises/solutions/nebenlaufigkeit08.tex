Der Konstruktor um \lstinline$Car$ muss um einen neuen Parameter erweitert werden, welcher eine
\lstinline$Barrier$ erwartet.
Diese \lstinline$Barrier$ muss in einem neuen Attribut gespeichert werden.
In der \lstinline$run()$-Methode muss vor der \lstinline$while$-Schleife die \lstinline$wait()$-Methode der
\lstinline$Barrier$ aufgerufen werden.

\lstinputlisting[language=Python,linerange={1-2,27-53}]{exercises/src/nebenlaufigkeit08.py}

Bevor alle \lstinline$Car$-Objekte erzeugt werden, muss eine \lstinline$Barrier$ initialisiert werden.
Ihr muss die Anzahl der erzeugten \lstinline$Cars$ mitgegeben werden.
Im Konstruktoraufruf von \lstinline$Car$ muss der neue Parameter erg�nzt werden.

\lstinputlisting[language=Python,linerange={1-2,55-62}]{exercises/src/nebenlaufigkeit08.py}
