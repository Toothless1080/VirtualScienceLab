Der selbst erstellten Klasse muss ein Konstruktor hinzugef�gt werden, der einen
Namen entgegen nimmt.
Dieser Name sollte in einem Attribut abgespeichert werden.
In der Schleife, welche in der \lstinline$run()$-Methode ausgef�hrt wird, muss nun zus�tzlich der
Name ausgegeben werden.

\lstinputlisting[language=Python,linerange={1-2,7-16}]{exercises/src/nebenlaufigkeit02.py}

Es m�ssen nun 100 Threads, denen unterschiedliche Namen �bergeben werden,
erzeugt und gestartet werden.

\lstinputlisting[language=Python,linerange={1-2,18-21}]{exercises/src/nebenlaufigkeit02.py}

Wird das Programm mehrmals ausgef�hrt, �ndert sich die Reihenfolge, in der die Threads die
Ausgaben t�tigen.