\begin{enumerate}
\item Der Konstruktor von \lstinline$Queue$ kann mit dem Wert 20 aufgerufen werden, um die Gr��e zu
begrenzen.
Anschlie�en kann durch Iterieren in einer \lstinline$for$-Schleife der \lstinline$Queue$ bef�llt werden.

\item Bei der Erzeugung der Threads ist es wichtig, die D�mon-Eigenschaft zu setzen.
Andernfalls beendet sich das Programm nicht, sobald sich der Main-Thread beendet.
Zur Implementierung der Threads ist es ausreichend in einer \lstinline$while$-Schleife ein Element
aus dem �bergebenen \lstinline$Queue$ zu lesen und das Quadrat auszugeben.
Es ist wichtig, \lstinline$task_done()$ nach der Ausgabe aufzurufen, da ansonsten der Main-Thread 
blockiert bleibt.

\item Wurden die Threads gestartet, muss \lstinline$join()$ auf der \lstinline$Queue$ im Main-Thread
aufgerufen werden.
Nach dem Aufruf kann eine weitere Ausgabe get�tig werden, um das Geschehen in der Konsole besser
nachzuvollziehen.
\end{enumerate}

\lstinputlisting[language=Python,linerange={1-2,7-25}]{exercises/src/nebenlaufigkeit10.py}

