Hierzu wird zuerst eine \lstinline$isPrim()$-Methode implementiert, welche 1 zur�ck gibt, falls die ihr
�bergebene Zahl eine Primzahl ist, andernfalls wird 0 zur�ck gegeben.

\lstinputlisting[language=Python,linerange={1-2,8-18}]{exercises/src/nebenlaufigkeit12.py}

Es wird nun ein \lstinline$Pool$ mit 10 Arbeiterprozessen erzeugt. 
Per \lstinline$imap()$ wird dem \lstinline$Pool$ die auszuf�hrende Methode, \lstinline$isPrim()$, und 
ein Iterable-Objekt, welche die Zahlen 1 bis 10 Millionen enth�lt �bergeben.
Um die Anzahl der Primzahlen zu erhalten kann das Ergebnis von \lstinline$imap()$ an die
\lstinline$sum()$-Methode �bergeben werden.

\lstinputlisting[language=Python,linerange={1-2,20-30}]{exercises/src/nebenlaufigkeit12.py}
